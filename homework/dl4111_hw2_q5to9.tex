\documentclass[11pt]{article}
\usepackage{fullpage}
\usepackage{amsmath,amsfonts,amsthm,amssymb}
\usepackage{url}
\usepackage[demo]{graphicx}
\usepackage{caption} 
\usepackage{algpseudocode}
\usepackage{bbm}
\usepackage{float}
\usepackage{framed}
\usepackage{enumerate}
\usepackage{color}
\usepackage{mathtools}
\usepackage[colorlinks=true, linkcolor=red, urlcolor=blue, citecolor=blue]{hyperref}

\DeclareMathOperator*{\E}{\mathbb{E}}
\let\Pr\relax
\DeclareMathOperator*{\Pr}{\mathbb{P}}
\DeclareMathOperator*{\R}{\mathbb{E}}

\topmargin 0pt
\advance \topmargin by -\headheight
\advance \topmargin by -\headsep
\textheight 8.9in
\oddsidemargin 0pt
\evensidemargin \oddsidemargin
\marginparwidth 0.5in
\textwidth 6.5in

\parindent 0in
\parskip 1.5ex

\newcommand{\homework}[2]{
	\noindent
	\begin{center}
		\framebox{
			\vbox{
				\hbox to 6.50in { {\bf NYU Computer Science Bridge to Tandon Course} \hfill Winter 2021 }
				\vspace{4mm}
				\hbox to 6.50in { {\Large \hfill Homework 2  \hfill} }
				\vspace{2mm}
				\hbox to 6.50in { {Name: Danrong Li \hfill} NetID: dl4111}

			}
		}
	\end{center}
	\vspace*{4mm}
}

\begin{document}
	
	\homework{1}{Danrong Li}
	\section*{Question 5}
	\textbf{A.} Solve the following questions from the Discrete Math zyBook:
	\medskip
	
	\textbf{1. b. }
	\begin{center}
	$p\xrightarrow[]{}(q\wedge r)$
	
	$\neg q$
	
	$\neg p$
	\end{center}
	
	\begin{center}
    \begin{tabular}{||c c c||} 
    \hline
    \# & statement & justification \\ [0.5ex] 
    \hline\hline
    1 & $\neg p\vee (q\wedge r)$ & Hypothesis \\ 
    \hline
    2 & $(\neg p\vee q)\wedge (\neg p\vee r)$ & Distributive laws 1 \\
    \hline
    3 & $\neg p\vee q$ & Simplification 2 \\
    \hline
    4 & $q\vee \neg p$ & Commutative laws 3 \\
    \hline
    5 & $\neg q$ & Hypothesis \\
    \hline
    6 & $\neg p$ & Disjunctive syllogism 4,5 \\
    \hline
    \end{tabular}
    \end{center}
	
	\textbf{1. e. }
	\begin{center}
	$p\vee q$
	
	$\neg p\vee r$
	
	$\neg q$
	
	$r$
	\end{center}
	
	\begin{center}
    \begin{tabular}{||c c c||} 
    \hline
    \# & statement & justification \\ [0.5ex] 
    \hline\hline
    1 & $p\vee q$ & Hypothesis \\ 
    \hline
    2 & $q\vee p$ & Commutative laws 1 \\
    \hline
    3 & $\neg q$ & Hypothesis \\
    \hline
    4 & $p$ & Disjunctive syllogism 2,3 \\
    \hline
    5 & $\neg p\vee r$ & Hypothesis \\
    \hline
    6 & $\neg\neg p$ & Double negation law 4 \\
    \hline
    7 & $r$ & Disjunctive syllogism 5,6 \\
    \hline
    \end{tabular}
    \end{center}
	
	\textbf{2. c. }
	\begin{center}
	$p\vee q$
	
	$\neg p$
	
	$q$
	\end{center}
	
	\begin{center}
    \begin{tabular}{||c c c||} 
    \hline
    \# & statement & justification \\ [0.5ex] 
    \hline\hline
    1 & $\neg p$ & Hypothesis \\ 
    \hline
    2 & $\neg p\vee {\rm F}$ & Addition 1 \\
    \hline
    3 & $p\vee q$ & Hypothesis \\
    \hline
    4 & $q\vee {\rm F}$ & Resolution 3,2 \\
    \hline
    5 & $q$ & Identity laws 4 \\
    \hline
    \end{tabular}
    \end{center}
    
    \textbf{3. c. }
	
	j: I will get a job.
	
	c: I will buy a new car.
	
	h: I will buy a new house.
	
	\begin{center}
	$(c\wedge h)\xrightarrow[]{}j$
	
	$\neg j$
	
	$\neg c$
	\end{center}
	
	\begin{center}
    \begin{tabular}{||c c c c c c||} 
    \hline
    c & h & j & $(c\wedge h)\xrightarrow[]{}j$ & $\neg j$ & $\neg c$\\ [0.5ex] 
    \hline\hline
    T & T & T & T & F & F \\ 
    \hline
    T & T & F & F & T & F \\ 
    \hline
    T & F & T & T & F & F \\ 
    \hline
    T & F & F & T & T & F \\ 
    \hline
    F & T & T & T & F & T \\ 
    \hline
    F & T & F & T & T & T \\ 
    \hline
    F & F & T & T & F & T \\ 
    \hline
    F & F & F & T & T & T \\ 
    \hline
    \end{tabular}
    \end{center}
    
    \textbf{The argument is not valid.}
    When c is True, h and j are False, the hypothesises are both True and the conclusion is False. Thus, the argument is invalid. 
	
	\vspace{4mm}
	\textbf{3. d.}
	
	j: I will get a job.
	
	c: I will buy a new car.
	
	h: I will buy a new house.
	
	\begin{center}
	$(c\wedge h)\xrightarrow[]{}j$
	
	$\neg j$
	
	$h$
	
	$\neg c$
	\end{center}
	
	\textbf{The argument is valid.}
	\begin{center}
    \begin{tabular}{||c c c||} 
    \hline
    \# & statement & justification \\ [0.5ex] 
    \hline\hline
    1 & $(c\wedge h)\xrightarrow[]{}j$ & Hypothesis \\ 
    \hline
    2 & $\neg j$ & Hypothesis \\
    \hline
    3 & $\neg (c\wedge h)$ & Modus tollens 2,1 \\
    \hline
    4 & $(\neg c)\vee (\neg h)$ & De Morgan's laws 3 \\
    \hline
    5 & $(\neg h)\vee (\neg c)$ & Commutative laws 4 \\
    \hline
    6 & $h$ & Hypothesis \\
    \hline
    7 & $\neg \neg h$ & Double negation 6 \\
    \hline
    8 & $\neg c$ & Disjunctive syllogism 5,7 \\
    \hline
    \end{tabular}
    \end{center}
	

	\vspace{4mm}
	\textbf{B.} Solve the following questions from the Discrete Math zyBook:
	\medskip
	
	\textbf{1. b.}
	
	\begin{center}
	$\exists_x (P(x)\vee Q(x))$
	
	$\exists_x \neg Q(x)$
	
	$\exists_x P(x)$

	\end{center}
	
	\begin{center}
	\begin{tabular}{||c c c||} 
    \hline
     & P & Q \\ [0.5ex] 
    \hline\hline
    a & F & F \\ 
    \hline
    b & F & T \\
    \hline
    \end{tabular}
	\end{center}
	
	\textbf{Given argument is invalid.}
	Since the values over the domain {a,b} make the hypothesises to be True and make the conclusion to be False, the given argument is invalid. 
	
	\vspace{2mm}
	\textbf{2. d.}
	
	M(x): x missed class.
	
	D(x): x got a detention.
	
	\begin{center}
	$\forall_x (M(x)\xrightarrow[]{}D(x))$
	
	Penelope is a student in class
	
	$\neg M(Penelope)$
	
	$\neg D(Penelope)$
	\end{center}
	
	\textbf{The argument is not valid.}
	Consider the situation that Penelope is a student in class: She missed a class but she got a detention for other reasons. In this case, the hypothesises would all be True, but the conclusion that Penelope did not get detention is False.
	
	\vspace{2mm}
	\textbf{2. e.}
	
	M(x): x missed a class.
	
	D(x): x got a detention.
	
	A(x): x received an A.
	
	\begin{center}
	$\forall_x(((M(x))\vee (D(x)))\xrightarrow[]{}(\neg A(x)))$
	
	Penelope is a student in class
	
	$A(Penelope)$
	
	$\neg D(Penelope)$
	
	\end{center}
	\textbf{This is valid.}
	
	\begin{center}
    \begin{tabular}{||c c c||} 
    \hline
    \# & statement & justification \\ [0.5ex] 
    \hline\hline
    1 & Penelope is a particular student in class & Hypothesis \\ 
    \hline
    2 & $\forall_x (((M(x))\vee (D(x)))\xrightarrow[]{}(\neg A(x)))$ & Hypothesis \\
    \hline
    3 & $((M(Penelope))\vee (D(Penelope)))\xrightarrow[]{}(\neg A(Penelope))$ & Universal instantiation 1,2 \\
    \hline
    4 & $A(Penelope)$ & Hypothesis \\
    \hline
    5 & $\neg \neg A(Penelope)$ & Double negation law 4 \\
    \hline
    6 & $\neg (M(Penelope)\vee D(Penelope))$ & Modus tollens 5,3 \\
    \hline
    7 & $\neg M(P)\wedge \neg D(P)$ & De Morgan's law 6 \\
    \hline
    8 & $\neg D(Penelope)$ & Simplification 7 \\
    \hline
    \end{tabular}
    \end{center}
	
	
	
	\newpage
	\section*{Question 6}
	
	\textbf{d.}
	\textbf{Proof.}
	
	Direct proof. Assume that m and n are two odd integers. We will show that m*n is also an odd integer. Since m, n are odd, m = 2*k+1, n = 2*j+1 for some integers k, j. Plug the expression for m, n into m*n:
	
	m*n = (2*k+1)*(2*j+1) = 2*(2kj+k+j) + 1
	
	Since k, j are integers, then (2kj+k+j) is also an integer. Since m*n = 2*c + 1, where c = 2kj+k+j is an integer, then m*n is odd. Thus, the product of two odd integers is an odd integer. \textbullet 
	
	\vspace{2mm}
	\textbf{c.}
	\textbf{proof.}
	
	Direct proof. Assume x is a real number and x $\leq$ 3, we will show that $12-7x+x^2 \geq 0$. Since $x\leq 3$, then we have 
	
	\newpage
	\section*{Question 7}
	\textbf{d.}
	
	\vspace{2mm}
	\textbf{f.}
	
	\vspace{2mm}
	\textbf{g.}
	
	\vspace{2mm}
	\textbf{l.}
	
	
	\newpage
	\section*{Question 8}
	
	\textbf{c.}
	
	\vspace{2mm}
	\textbf{e.}
	
	\newpage
	\section*{Question 9}
	
	\textbf{c.}
	
	

	

\end{document}