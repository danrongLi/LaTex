\documentclass[11pt]{article}
\usepackage{fullpage}
\usepackage{amsmath,amsfonts,amsthm,amssymb}
\usepackage{url}
\usepackage[demo]{graphicx}
\usepackage{caption} 
\usepackage{algpseudocode}
\usepackage{bbm}
\usepackage{float}
\usepackage{framed}
\usepackage{enumerate}
\usepackage{color}
\usepackage{mathtools}
\usepackage[colorlinks=true, linkcolor=red, urlcolor=blue, citecolor=blue]{hyperref}

\usepackage[utf8]{inputenc}
\usepackage[english]{babel}
\renewcommand\qedsymbol{$\blacksquare$}

\usepackage{amsthm}

\DeclareMathOperator*{\E}{\mathbb{E}}
\let\Pr\relax
\DeclareMathOperator*{\Pr}{\mathbb{P}}
\DeclareMathOperator*{\R}{\mathbb{E}}

\topmargin 0pt
\advance \topmargin by -\headheight
\advance \topmargin by -\headsep
\textheight 8.9in
\oddsidemargin 0pt
\evensidemargin \oddsidemargin
\marginparwidth 0.5in
\textwidth 6.5in

\parindent 0in
\parskip 1.5ex

\newcommand{\homework}[2]{
	\noindent
	\begin{center}
		\framebox{
			\vbox{
				\hbox to 6.50in { {\bf NYU Computer Science Bridge to Tandon Course} \hfill Winter 2021 }
				\vspace{4mm}
				\hbox to 6.50in { {\Large \hfill Homework 5  \hfill} }
				\vspace{2mm}
				\hbox to 6.50in { {Name: Danrong Li \hfill} NetID: dl4111}

			}
		}
	\end{center}
	\vspace*{4mm}
}

\begin{document}
	
	\homework{1}{Danrong Li}
	\section*{Question 3}
	
	Solve the following questions from the Discrete Math zyBook:
	
	\textbf{1. Exercise 4.1.3, sections b, c}
	
	\textbf{b. Not a well-defined function}
	
	If x = 2 or x = -2, the denominator would be 0. 
	Then f(x) would not be a real number.
	
	\textbf{c. A well-defined function}
	
	The range is $[0, \infty]$
	
	\vspace{10mm}
	\textbf{2. Exercise 4.1.5, sections b, d, h, i ,l}
	
	\textbf{b.}
	$\{$ 4, 9, 16, 25 $\}$
	
	\textbf{d.}
	$\{$ 0, 1, 2, 3, 4, 5 $\}$
	
	\textbf{h.}
	$\{ (1,1), (2,1), (3,1), (1,2), (2,2), (3,2), (1,3), (2,3), (3,3) \}$
	
	\textbf{i.}
	$\{ (1,2), (1,3), (1,4), (2,2), (2,3), (2,4), (3,2), (3,3), (3,4) \}$
	
	\textbf{l.}
	$\{\varnothing,\{2\}, \{3\}, \{2,3\}\}$ 
	
	
	\newpage
	\section*{Question 4}
	
	A. Solve the following questions from the Discrete Math zyBook:
	
	\textbf{1. Exercise 4.2.2, sections c, g, k}
	
	\textbf{c. It's one-to-one, but not onto}
	
	Let y be a integer which is not a perfect cube, for instance we have y = 2. In this case, there is no $x\in\mathbb{Z}$ such that $x^3=y$
	
	\textbf{g. It's one-to-one, but not onto}
	
	Since 2*y always produces even results, there is no element in the domain that would lead to the element of (1,1) in the target when $x\in\mathbb{Z}$ and $y\in\mathbb{Z}$.
	
	\textbf{k. It's one-to-one, but not onto}
	
	There is no element in the domain that would lead to the element of 1 (a positive integer) in the target. The smallest element in range is 3 when x = 1 and y = 1.
	
	\vspace{10mm}
	\textbf{2. Exercise 4.2.4, sections b, c, d, g}
	
	\textbf{b. It's neither one-to-one nor onto}
	
	Since f(110) = f(010) = 110, it is not one-to-one. 
	
	Since the element of (0,0,0) in the target would not be mapped by any element from the domain, it is not onto.
	
	\textbf{c. It's one-to-one and onto}
	
	\textbf{d. It's one-to-one, but not onto}
	
	The element of 0001 in the target would not be mapped by any element from the domain.
	
	\textbf{g. It's neither one-to-one nor onto}
	
	Since $f(\{1,2,3\})=f(\{2,3\})=\{2,3\}$, it is not one-to-one.
	
	Since $\{1\}$ in the target would never be mapped from any element in the domain, it is not onto.
	
	\vspace{10mm}
	B. Give an example of a function from the set of integers to the set of positive integers that is:
	
	\textbf{1. one-to-one, but not onto}
	
    \begin{equation*}
    f\colon\mathbb{Z}\rightarrow\mathbb{Z^+}, f(x) = 
    \begin{cases}
    2*|x| + 3, \text{ if } x\leq 0\\
    2*x, \text{ if } x > 0
    \end{cases}
    \end{equation*}
	
	\textbf{2. onto, but not one-to-one}
	
	\begin{equation*}
	    f\colon\mathbb{Z}\rightarrow\mathbb{Z^+}, f(x) = |x| + 1
	\end{equation*}
	
	\textbf{3. one-to-one and onto}
	
	\begin{equation*}
	    f\colon\mathbb{Z}\rightarrow\mathbb{Z^+}, f(x) = 
	    \begin{cases}
	        2*|x|+1, \text{ if } x\leq 0\\
	        2*x, \text{ if } x > 0
	    \end{cases}
	\end{equation*}
	
	\textbf{4. neither one-to-one nor onto}
	
	\begin{equation*}
	    f\colon\mathbb{Z}\rightarrow\mathbb{Z^+}, f(x) = 1
	\end{equation*}
	
	\newpage
	\section*{Question 5}
	
	Solve the following questions from the Discrete Math zyBook:
	
	\textbf{1. Exercise 4.3.2, sections c, d, g, i}
	
	\textbf{c.}
	$f^-^1(x)=(x-3)/2$
	
	\textbf{d.}
	It's onto, but not one-to-one. So there is no well-defined inverse.
	
	$f(\{1\})=f(\{2\})=1$, so it is not one-to-one.
	
	\textbf{g.}
	$f^{-1}(x)=f(x)=\{0,1\}^3$
	
	\textbf{i.}
	$f^{-1}(x,y)=(x-5,y+2)$
	
	\vspace{10mm}
	\textbf{2. Exercise 4.4.8, sections c, d}
	
	\textbf{c.}
	$f\circ h$ = $f(h(x))$ = $2*(x^2+1)+3$ = $2*x^2+5$
	
	\textbf{d.}
	$h\circ f$ = $h(f(x))$ = $(2*x+3)^2+1$ = $4*x^2+12*x+10$
	
	\vspace{10mm}
	\textbf{3. Exercise 4.4.2, sections b-d}
	
	\textbf{b.}
	$f\circ h(52)$ = $(\lceil 52/5\rceil)^2$ = $11^2$ = 121
	
	\textbf{c.}
	$g\circ h\circ f(4)$ = $2^{\lceil 16/5\rceil}$ = $2^4$ = 16
	
	\textbf{d.}
	$h\circ f$ = $h(f(x))$ = $\lceil x^2/5\rceil$
	
	\vspace{10mm}
	\textbf{4. Exercise 4.4.6, sections c-e}
	
	\textbf{c.}
	$h\circ f(010)$ = $h(110)$ = 111
	
	\textbf{d.}
	$h\circ f$ = $h(f(x))$
	
	\textbf{So the range is:} $\{101,111\}$
	
	\textbf{e.}
	$g\circ f$ = $g(f(x))$
	
	\textbf{So the range is:} $\{001,011,101,111\}$
	
	\vspace{10mm}
	\textbf{5. Extra Credit: Exercise 4.4.4, sections c, d}
	
	\textbf{c. It's not possible}
	
	f: $X\rightarrow Y$, g: $Y\rightarrow Z$
	
	If f is not one-to-one, then it means that $\exists_{x1, x2 \in X}: (x1\neq x2)\wedge (f(x1)=f(x2))$
	
	And this leads to 
	$\exists_{x1,x2\in X}(x1\neq x2)\wedge(g(f(x1))\neq g(f(x2)))$
	
	So we get $g\circ f$ is not one-to-one. And the answer is that it is not possible.
	
	\textbf{d. It is possible}

	We can see from the diagram below:
	
\end{document}