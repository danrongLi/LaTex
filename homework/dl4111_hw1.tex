\documentclass[11pt]{article}
\usepackage{fullpage}
\usepackage{amsmath,amsfonts,amsthm,amssymb}
\usepackage{url}
\usepackage[demo]{graphicx}
\usepackage{caption} 
\usepackage{algpseudocode}
\usepackage{bbm}
\usepackage{float}
\usepackage{framed}
\usepackage{enumerate}
\usepackage{color}
\usepackage{mathtools}
\usepackage[colorlinks=true, linkcolor=red, urlcolor=blue, citecolor=blue]{hyperref}

\DeclareMathOperator*{\E}{\mathbb{E}}
\let\Pr\relax
\DeclareMathOperator*{\Pr}{\mathbb{P}}
\DeclareMathOperator*{\R}{\mathbb{E}}

\topmargin 0pt
\advance \topmargin by -\headheight
\advance \topmargin by -\headsep
\textheight 8.9in
\oddsidemargin 0pt
\evensidemargin \oddsidemargin
\marginparwidth 0.5in
\textwidth 6.5in

\parindent 0in
\parskip 1.5ex

\newcommand{\homework}[2]{
	\noindent
	\begin{center}
		\framebox{
			\vbox{
				\hbox to 6.50in { {\bf NYU Computer Science Bridge to Tandon Course} \hfill Winter 2021 }
				\vspace{4mm}
				\hbox to 6.50in { {\Large \hfill Homework #1  \hfill} }
				\vspace{2mm}
				\hbox to 6.50in { {Name: Danrong Li \hfill} NetID: dl4111}

			}
		}
	\end{center}
	\vspace*{4mm}
}

\begin{document}
	
	\homework{1}{Danrong Li}
	\section*{Question 1}
	\textbf{A.} Convert the following numbers to their decimal representation. Show your work.
	\medskip
	
	1. $10011011_2$ = $2^7+2^4+2^3+2^1+2^0$ = \textbf{155}
	
	2. $456_7$ = $4*7^2+5*7^1+6*7^0$ = \textbf{237}
	
	3. $38{\rm A}_{16}$ = $3*16^2+8*16^1+10*16^0$ = \textbf{906}
	
	4. $2214_5$ = $2*5^3+2*5^2+1*5^1+4*5^0$ = \textbf{309}

	\vspace{3mm}
	\textbf{B.} Convert the following numbers to their binary representation:
	\medskip
	
	1. $69_{10}$ = $64+4+1$ = $2^6+2^2+2^0$ = $\mathbf{(1000101)_2}$
	
	2. $485_{10}$ = $256+128+64+32+4+1$ = $2^8+2^7+2^6+2^5+2^2+2^0$ = $\mathbf{(111100101)_2}$
	
	3. From the Hex Digit and 4 bit binary Conversion Table:
	
	\hspace{4mm} 6 = 0110; D = 1101; 1 = 0001; A = 1010; 
	
	\hspace{4mm} Thus,
	$6{\rm D}1{\rm A}_{16}$ = $\mathbf{(0110110100011010)_2}$
	
	\vspace{3mm}
	\textbf{C.} Convert the following numbers to their hexadecimal representation:
	\medskip
	
	1. From the Hex Digit and 4 bit binary Conversion Table:
	
	\hspace{4mm} 6 = 0110; B = 1011; 
	
	\hspace{4mm} Thus,
	$1101011_2$ = $\mathbf{(6\rm \mathbf{B)_{16}}}$
	
	2. $895_{10}$ = $256*3+16*7+1*15$ = $16^2*3+16^1*7+16^0*15$ = $\mathbf{(37\rm \mathbf{F)_{16}}}$
	
	\newpage
	\section*{Question 2}
    Solve the following, do all calculation in the given base. Show your work.
	\medskip
	
	1. $7566_8+4515_8$ = $\mathbf{(14303)_8}$
	
	Work process: 
	\vspace{30mm}

	2. $10110011_2+1101_2$ = $\mathbf{(11000000)_2}$
	
	Work process: 
	\vspace{30mm}	

	3. $7{\rm A}66_{16}+45{\rm C}5_{16}$ = $\mathbf{(C02\rm \mathbf{B)_{16}}}$

	Work process: 
	\vspace{30mm}
	
	4. $3022_5-2433_5$ = $\mathbf{(34)_5}$
	
	Work process: 
	\vspace{30mm}
	
	\newpage
	\section*{Question 3}
	\textbf{A.} Convert the following numbers to their 8-bits two's complement representation. Show your work.
	\medskip
	
	1. $124_{10}$ = $64+32+16+8+4$ = $2^6+2^5+2^4+2^3+2^2$ = $(1111100)_2$ =  $\mathbf{01111100_{\rm \mathbf{8\ bit\ 2's\ comp}}}$
	
	2. $-124_{10}$ = $\mathbf{10000100_{\rm \mathbf{8\ bit\ 2's\ comp}}}$

	Work process: 
	\vspace{30mm}
	
	3. $109_{10}$ = $64+32+8+4+1$ = $2^6+2^5+2^3+2^2+2^0$ = $(1101101)_2$ = $\mathbf{01101101_{\rm \mathbf{8\ bit\ 2's\ comp}}}$

	
	4. $-79_{10}$ = $\mathbf{10110001_{\rm \mathbf{8\ bit\ 2's\ comp}}}$

	Work process: 
	
	$79_{10}$ = $64+8+4+2+1$ = $2^6+2^3+2^2+2^1+2^0$ = $(1001111)_2$ = $01001111_{\rm 8\ bit\ 2's\ comp}$
	\vspace{25mm}

	\vspace{3mm}
	\textbf{B.} Convert the following numbers (represented as 8-bit two's complement) to their decimal representation. Show your work.
	\medskip
	
	1. $00011110_{\rm 8\ bit\ 2's\ comp}$ = $2^4+2^3+2^2+2^1$ = $\mathbf{30}$
	
	2. $11100110_{\rm 8\ bit\ 2's\ comp}$ = $\mathbf{-26}$
	
	Work process:
	\vspace{25mm}
	
	$2^4+2^3+2^1$ = $26$\hspace{3mm}
	Thus, $11100110_{\rm 8\ bit\ 2's\ comp}$ = $\mathbf{-26}$
	
	3. $00101101_{\rm 8\ bit\ 2's\ comp}$ = $2^5+2^3+2^2+2^0$ = $\mathbf{45}$
	
	4. $10011110_{\rm 8\ bit\ 2's\ comp}$ = $\mathbf{-98}$
	
	Work process: (next page)
	
	Work process:
	\vspace{30mm}
	
	$2^6+2^5+2^1$ = $98$\hspace{3mm}
	Thus, $10011110_{\rm 8\ bit\ 2's\ comp}$ = $\mathbf{-98}$
	
	\newpage
	\section*{Question 4}
	Solve the following questions from the Discrete Math zyBook:
	
	\textbf{1.} Exercise 1.2.4, section b,c
	\medskip
	
	b. 
	\begin{center}
    \begin{tabular}{||c c c||} 
    \hline
    p & q &$ \neg (p\vee q)$ \\ [0.5ex] 
    \hline\hline
    T & T & F \\ 
    \hline
    T & F & F \\
    \hline
    F & T & F \\
    \hline
    F & F & T \\
    \hline
    \end{tabular}
    \end{center}
    
    c. 
	\begin{center}
    \begin{tabular}{||c c c c||} 
    \hline
    p & q & r & $r \vee (p\wedge \neg q)$ \\ [0.5ex] 
    \hline\hline
    T & T & T & T \\ 
    \hline
    T & T & F & F \\
    \hline
    T & F & T & T \\
    \hline
    T & F & F & T \\
    \hline
    F & T & T & T \\ 
    \hline
    F & T & F & F \\
    \hline
    F & F & T & T \\
    \hline
    F & F & F & F \\
    \hline
    \end{tabular}
    \end{center}
    
	\vspace{3mm}
	\textbf{2.} Exercise 1.3.4, section b,d
	\medskip
	
	b.
	\begin{center}
    \begin{tabular}{||c c c||} 
    \hline
    p & q & $(p\xrightarrow[]{}q)\xrightarrow[]{}(q\xrightarrow[]{}p)$ \\ [0.5ex] 
    \hline\hline
    T & T & T \\ 
    \hline
    T & F & T \\
    \hline
    F & T & F \\
    \hline
    F & F & T \\
    \hline
    \end{tabular}
    \end{center}	
    
    d.
	\begin{center}
    \begin{tabular}{||c c c||} 
    \hline
    p & q & $(p\xleftrightarrow[]{}q)\bigoplus(p\xleftrightarrow[]{}\neg q)$ \\ [0.5ex] 
    \hline\hline
    T & T & T \\ 
    \hline
    T & F & T \\
    \hline
    F & T & T \\
    \hline
    F & F & T \\
    \hline
    \end{tabular}
    \end{center}	
	

	\newpage
	\section*{Question 5}
	Solve the following questions from the Discrete Math zyBook:
	
	\textbf{1.} Exercise 1.2.7, section b,c
	\medskip
	
	b. $(B \wedge D) \vee (B \wedge M) \vee (D \wedge M)$
	
	c. $B \vee (D \wedge M)$

	\vspace{3mm}
	\textbf{2.} Exercise 1.3.7, section b-e
	\medskip
	
	b. $(s \vee y)\xrightarrow[]{}p$
	
	c. $p \xrightarrow[]{}y$
	
	d. $p\xleftrightarrow[]{}(s \wedge y)$
	
	e. $p\xrightarrow[]{}(s \vee y)$
	

	\vspace{3mm}
	\textbf{3.} Exercise 1.3.9, section c,d
	\medskip
	
	c. $c\xrightarrow[]{}p$
	
	d. $c\xrightarrow[]{}p$
	
	\newpage
	\section*{Question 6}
	Solve the following questions from the Discrete Math zyBook:
	
	\textbf{1.} Exercise 1.3.6, section b-d
	\medskip
	
	b. \textbf{If Joe is eligible for the honors program, then Joe maintains a B average.}
	
	c. \textbf{If Rajiv can go on the roller coaster, then he is at least four feet tall.}
	
	d. \textbf{If Rajiv is at least four feet tall, then he can go on the roller coaster.}
	
	\vspace{3mm}
	\textbf{2.} Exercise 1.3.10, section c-f
	\medskip
	
	c. \textbf{False}
	
	With r being unknown, the left-hand-side would always be True and the right-hand-side would always be False. Thus, we get $\rm T\xleftrightarrow[]{}\rm F$, which is \textbf{False}.
	
	d. \textbf{Unknown}
	
	If r is True, we get $\rm T\xleftrightarrow[]{}\rm F$, which is False.
	
	If r is False, we get $\rm F\xleftrightarrow[]{}\rm F$, which is True.
	
	Since we get two results depending on r, we get \textbf{Unknown} for this question.
	
	e. \textbf{Unknown}
	
	If r is True, we get $\rm T\xrightarrow[]{}\rm T$, which is True.
	
	If r is False, we get $\rm T\xrightarrow[]{}\rm F$, which is False.
	
	Since we get two results depending on r, we get \textbf{Unknown} for this question.
	
	f. \textbf{True}
	
	If r is True, we get $\rm F\xrightarrow[]{}\rm F$, which is True.
	
	If r is False, we get $\rm F\xrightarrow[]{}\rm T$, which is True.
	
	Since we get True in both cases, we get \textbf{True} for this question.
	

	\newpage
	\section*{Question 7}
	Solve the following questions from the Discrete Math zyBook:
	
	Exercise 1.4.5, section b-d
	\medskip
	
	b. \textbf{not logically equivalent}
	\begin{center}
    \begin{tabular}{||c c c c c||} 
    \hline
    j & l & r & $\neg j\xrightarrow[]{}(l\vee r)$ & $(r\wedge \neg l)\xrightarrow[]{}j$\\ [0.5ex] 
    \hline\hline
    T & T & T & T & T\\ 
    \hline
    T & T & F & T & T\\ 
    \hline
    T & F & T & T & T\\ 
    \hline
    T & F & F & T & T\\ 
    \hline
    F & T & T & T & T\\ 
    \hline
    F & T & F & T & T\\ 
    \hline
    F & F & T & T & F\\ 
    \hline
    F & F & F & F & T\\ 
    \hline
    \end{tabular}
    \end{center}

    c. \textbf{not logically equivalent}
    \begin{center}
    \begin{tabular}{||c c c c||} 
    \hline
    j & l & $j\xrightarrow[]{}\neg l$ & $\neg j\xrightarrow[]{}l$ \\ [0.5ex] 
    \hline\hline
    T & T & F & T \\ 
    \hline
    T & F & T & T \\
    \hline
    F & T & T & T \\
    \hline
    F & F & T & F \\
    \hline
    \end{tabular}
    \end{center}
    
    d. \textbf{not logically equivalent}
    \begin{center}
    \begin{tabular}{||c c c c c||} 
    \hline
    j & l & r & $(r\vee \neg l)\xrightarrow[]{}j$ & $j\xrightarrow[]{}(r\wedge \neg l)$\\ [0.5ex] 
    \hline\hline
    T & T & T & T & F\\ 
    \hline
    T & T & F & T & F\\ 
    \hline
    T & F & T & T & T\\ 
    \hline
    T & F & F & T & F\\ 
    \hline
    F & T & T & F & T\\ 
    \hline
    F & T & F & T & T\\ 
    \hline
    F & F & T & F & T\\ 
    \hline
    F & F & F & F & T\\ 
    \hline
    \end{tabular}
    \end{center}
    
	\newpage
	\section*{Question 8}
	Solve the following questions from the Discrete Math zyBook:
	
	\textbf{1.} Exercise 1.5.2, section c,f,i
	\medskip
	
	\textbf{c.}
	$(p\xrightarrow[]{}q)\wedge (p\xrightarrow[]{}r)$
	
	= $(\neg p\vee q)\wedge (\neg p\vee r)$\hspace{5mm} \textbf{Conditional identity}
	
	= $\neg p\vee (q\wedge r)$\hspace{18mm}\textbf{Distributive law}
	
	=$p\xrightarrow[]{}(q\wedge r)$\hspace{20mm}\textbf{Conditional identity}
	\vspace{2mm}
	
	\textbf{f.}
	$\neg (p\vee (\neg p\wedge q))$

	= $\neg p\wedge (\neg (\neg p\wedge q))$\hspace{9mm}\textbf{De Morgan's law}
	
	= $\neg p\wedge (\neg \neg p\vee \neg q)$\hspace{10mm}\textbf{De Morgan's law}
	
	= $\neg p\wedge (p\vee \neg q)$\hspace{15mm}\textbf{Double negation law}
	
	= $(\neg p\wedge p)\vee (\neg p \wedge \neg q)$\hspace{3mm}\textbf{Distributive law}
	
	= $(p\wedge \neg p)\vee (\neg p\wedge \neg q)$\hspace{3mm}\textbf{Commutative law}
	
	= $\rm F\vee (\neg p\wedge \neg q)$\hspace{14.5mm}\textbf{Complement law}
	
	= $(\neg p\wedge \neg q)\vee \rm F$\hspace{14mm}
	\textbf{Commutative law}
	
	= $\neg p\wedge \neg q$\hspace{25mm}\textbf{Identity law}
	\vspace{2mm}
	
	\textbf{i.}
	$(p\wedge q)\xrightarrow[]{}r$
	
	= $\neg (p\wedge q)\vee r$\hspace{19mm}\textbf{Conditional identity}
	
	= $(\neg p\vee \neg q)\vee r$\hspace{17mm}\textbf{De Morgan's law}
	
	= $(\neg q\vee \neg p)\vee r$\hspace{17mm}\textbf{Commutative law}
	
	= $\neg q\vee (\neg p\vee r)$\hspace{17mm}\textbf{Associative law}
	
	= $(\neg p\vee r)\vee \neg q$\hspace{17mm}\textbf{Commutative law}
	
	= $\neg (p\wedge \neg r)\vee \neg q$\hspace{15mm}\textbf{De Morgan's law}
	
	= $(p\wedge \neg r)\xrightarrow[]{}\neg q$\hspace{15.5mm}\textbf{Conditional identity}
	
	\vspace{3mm}
	\textbf{2.} Exercise 1.5.3, section c,d
	\medskip
	
	\textbf{c.}
	$\neg r\vee (\neg r\xrightarrow[]{}p)$

	= $\neg r\vee (\neg \neg r\vee p)$\hspace{15mm}\textbf{Conditional identity}
	
	= $\neg r\vee (r\vee p)$\hspace{20mm}\textbf{Double negation law}
	
	= $(\neg r\vee r)\vee p$\hspace{20mm}\textbf{Associative law}
	
	= $(r\vee \neg r)\vee p$\hspace{20mm}\textbf{Commutative law}
	
	= $\rm T\vee p$\hspace{30.5mm}\textbf{Complement law}
	
	= $p\vee \rm T$\hspace{31mm}\textbf{Commutative law}
	
	= $\rm T$\hspace{37.5mm}\textbf{Domination law}
	
	Thus, the statement is a \textbf{tautology}.
	\vspace{2mm}
	
	\textbf{d.}
    $\neg (p\xrightarrow[]{}q)\xrightarrow[]{}\neg q$
    
    = $\neg (\neg p\vee q)\xrightarrow[]{}\neg q$\hspace{16mm}\textbf{Conditional identity}
    
    = $(\neg p\vee q)\vee \neg q$\hspace{20mm}\textbf{Conditional identity}
	
	= $\neg p\vee (q\vee \neg q)$\hspace{20mm}\textbf{Associative law}
	
	= $\neg p\vee \rm T$\hspace{31mm}\textbf{Complement law}
	
	= $\rm T$\hspace{40mm}\textbf{Domination law}
	
	Thus, the statement is a \textbf{tautology}.
	
	\newpage
	\section*{Question 9}
	Solve the following questions from the Discrete Math zyBook:
	
	\textbf{1.} Exercise 1.6.3, section c,d
	\medskip
	
	\textbf{c.} $\exists_x (x=x^2)$
	
	\textbf{d.} $\forall_x (x\leq x^2)$
	
	
	\vspace{3mm}
	\textbf{2.} Exercise 1.7.4, section b-d
	\medskip
	
	\textbf{b.}
	$\forall_x (\neg S(x)\wedge W(x))$
	
	\textbf{c.}
	$\forall_x (S(x)\xrightarrow[]{}\neg W(x))$
	
	\textbf{d.}
	$\exists_x (S(x)\wedge W(x))$

	\newpage
	\section*{Question 10}
	Solve the following questions from the Discrete Math zyBook:
	
	\textbf{1.} Exercise 1.7.9, section c-i
	\medskip
	
	c. \textbf{True}
	
	Example: when x=a, we get $\rm F\xrightarrow[]{}\rm T$, which is \textbf{True}.
	
	d. \textbf{True}
	
	Example: when x=e
	
	e. \textbf{True}
	
	We get $\rm T\wedge \rm T$, which leads to \textbf{True}.
	
	f. \textbf{True}
	
	g. \textbf{False}
	
	Counter example: when x=c
	
	h. \textbf{True}
	
	i. \textbf{True}
	
	Example: when x=a
	
	\vspace{18mm}
	\textbf{2.} Exercise 1.9.2, section b-i
	\medskip
	
	b. \textbf{True}
	
	Example: Q(2,1) Q(2,2) Q(2,3)
	
	c. \textbf{True}
	
	Example: P(1,1) P(2,1) P(3,1)
	
	d. \textbf{False}
	
	All values for predicate S(x,y) are False.
	
	e. \textbf{False}
	
	Example: Q(1,1) Q(2,1) Q(3,1)
	
	f. \textbf{True}
	
	Example: P(1,1) P(2,1) P(3,1)
	
	g. \textbf{False}
	
	Example: P(1,2) P(2,2) P(3,2)
	
	h. \textbf{True}
	
	Example: Q(2,1) Q(2,2) Q(2,3)
	
	i. \textbf{True}
	
	All values for predicate $\neg$S(x,y) are True.
	
	\newpage
	\section*{Question 11}
	Solve the following questions from the Discrete Math zyBook:
	
	\textbf{1.} Exercise 1.10.4, section c-g
	\medskip
	
	\textbf{c.}
	$\exists_x \exists_y ((x+y)=(x*y))$
	
	\textbf{d.}
	$\forall_x \forall_y (((x>0)\wedge (y>0))\xrightarrow[]{}(x/y > 0))$
	
	\textbf{e.}
	$\forall_x (((x>0)\wedge (x<1))\xrightarrow[]{}(1/x > 1))$
	
	\textbf{f.}
	$\forall_x \exists_y (y<x)$
	
	\textbf{g.}
	$\forall_x \exists_y ((x\neq 0)\xrightarrow[]{}(x*y=1))$
	
	\vspace{3mm}
	\textbf{2.} Exercise 1.10.7, section c-f
	\medskip
	
	\textbf{c.}
	$\exists_x (N(x)\wedge D(x))$
	
	\textbf{d.}
	$\forall_x (D(x)\xrightarrow[]{}P(Sam,x))$
	
	\textbf{e.}
	$\exists_x \forall_y (N(x)\wedge P(x,y))$
	
	\textbf{f.}
	$\exists_x((N(x)\wedge D(x))\wedge \forall_y (((x\neq y)\wedge (N(y)))\xrightarrow[]{}\neg D(y)))$
	
	\vspace{3mm}
	\textbf{3.} Exercise 1.10.10, section c-f
	\medskip
	
	\textbf{c.}
	$\forall_x \exists_y ((y\neq Math101)\wedge T(x,y))$
	
	\textbf{d.}
	$\exists_x \forall_y ((y\neq Math101)\xrightarrow[]{}T(x,y))$
	
	\textbf{e.}
	$\forall_x \exists_y \exists_z ((x\neq Sam)\xrightarrow[]{}(T(x,y) \wedge (z\neq y)\wedge T(x,z)))$
	
	\textbf{f.}
	$\exists_y \exists_z \forall_p (T(Sam,y)\wedge (z\neq y)\wedge T(Sam,z)\wedge ((p\neq y)\wedge (p\neq z))\xrightarrow[]{}\neg T(Sam,p))$
	
	\newpage
	\section*{Question 12}
	Solve the following questions from the Discrete Math zyBook:
	
	\textbf{1.} Exercise 1.8.2, section b-e
	\medskip
	
	\textbf{b.}
	$\forall_x (D(x)\vee P(x))$
	
	\textbf{Negation:}
	$\neg \forall_x (D(x)\vee P(x))$
	
	De Morgan's law: $\exists_x \neg (D(x)\vee P(x))$
	
	De Morgan's law: $\exists_x (\neg D(x)\wedge \neg P(x))$
	
	\textbf{Translation:}
	There is a patient who has not given medication and not given placebo.
	\vspace{2mm}
	
	\textbf{c.}
	$\exists_x (D(x)\wedge M(x))$
	
	\textbf{Negation:}
	$\neg \exists_x (D(x)\wedge M(x))$
	
	De Morgan's law: $\forall_x \neg (D(x)\wedge M(x))$
	
	De Morgan's law: $\forall_x (\neg D(x)\vee \neg M(x))$
	
	\textbf{Translation:}
	Every patient was either not given medication or not had migraines or both.
	\vspace{2mm}
	
	\textbf{d.}
	$\forall_x (P(x)\xrightarrow[]{}M(x))$
	
	\textbf{Negation:}
	$\neg \forall_x (P(x)\xrightarrow[]{}M(x))$
	
	Conditional identity: $\neg \forall_x (\neg P(x)\vee M(x))$
	
	De Morgan's law: $\exists_x \neg (\neg P(x)\vee M(x))$
	
	De Morgan's law: $\exists_x (\neg \neg P(x)\wedge \neg M(x))$
	
	Double negation law: $\exists_x (P(x)\wedge \neg M(x))$
	
	\textbf{Translation:}
	There is a patient who was given placebo and not have migraines.
	\vspace{2mm}
	
	\textbf{e.}
	$\exists_x (M(x)\wedge P(x))$
	
	\textbf{Negation:}
	$\neg \exists_x (M(x)\wedge P(x))$
	
	De Morgan's law: $\forall_x \neg (M(x)\wedge P(x))$
	
	De Morgan's law: $\forall_x (\neg M(x)\vee \neg P(x))$
	
	\textbf{Translation:}
	Every patient was either not have migraines or not given placebo or both.
	
	
	\vspace{3mm}
	\textbf{2.} Exercise 1.9.4, section c-e
	\medskip
	
	\textbf{c.}
	$\exists_x \forall_y (P(x,y)\xrightarrow[]{}Q(x,y))$
	
	De Morgan's law: 
	$\forall_x \exists_y \neg (P(x,y)\xrightarrow[]{}Q(x,y))$
	
	Conditional identity:
	$\forall_x \exists_y \neg (\neg P(x,y)\vee Q(x,y))$
	
	De Morgan's law:
	$\forall_x \exists_y (\neg \neg P(x,y)\wedge \neg Q(x,y))$
	
	Double negation law:
	$(\forall_x \exists_y (P(x,y)\wedge \neg Q(x,y)))$
	
	\textbf{Thus, final answer: }
	$(\forall_x \exists_y (P(x,y)\wedge \neg Q(x,y)))$
	\vspace{2mm}
	
	\textbf{d.}
	$\exists_x \forall_y (P(x,y)\xleftrightarrow[]{}P(y,x))$
	
	De Morgan's law:
	$\forall_x \exists_y \neg (P(x,y)\xleftrightarrow[]{}P(y,x))$
	
	Conditional identity:
	$\forall_x \exists_y \neg ((P(x,y)\xrightarrow[]{}P(y,x))\wedge (P(y,x)\xrightarrow[]{}P(x,y))$
	
	Conditional identity:
	$\forall_x \exists_y \neg ((\neg P(x,y)\vee P(y,x))\wedge (\neg P(y,x)\vee P(x,y)))$
	
	De Morgan's law:
	$\forall_x \exists_y (\neg (\neg P(x,y)\vee P(y,x))\vee \neg(\neg P(y,x)\vee P(x,y)))$
	
	De Morgan's law:
	$\forall_x \exists_y ((\neg \neg P(x,y)\wedge \neg P(y,x))\vee (\neg \neg P(y,x)\wedge \neg P(x,y))))$
	
	Double negation law:
	$\forall_x \exists_y ((P(x,y)\wedge \neg P(y,x))\vee (P(y,x)\wedge \neg P(x,y)))$
	
	\textbf{Thus, final answer: }
	$\forall_x \exists_y ((P(x,y)\wedge \neg P(y,x))\vee (P(y,x)\wedge \neg P(x,y)))$
	\vspace{2mm}
	
	\textbf{e.}
	$\exists_x \exists_y P(x,y)\wedge \forall_x \forall_y Q(x,y)$
	
	De Morgan's law:
	$\forall_x \forall_y \neg P(x,y)\vee \exists_x \exists_y \neg Q(x,y)$
	
	\textbf{Thus, final answer: }
	$\forall_x \forall_y \neg P(x,y)\vee \exists_x \exists_y \neg Q(x,y)$

\end{document}