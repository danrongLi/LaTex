\documentclass[11pt]{article}
\usepackage{fullpage}
\usepackage{amsmath,amsfonts,amsthm,amssymb}
\usepackage{url}
\usepackage[demo]{graphicx}
\usepackage{caption} 
\usepackage{algpseudocode}
\usepackage{bbm}
\usepackage{float}
\usepackage{framed}
\usepackage{enumerate}
\usepackage{color}
\usepackage{mathtools}
\usepackage[colorlinks=true, linkcolor=red, urlcolor=blue, citecolor=blue]{hyperref}

\usepackage[utf8]{inputenc}
\usepackage[english]{babel}
\renewcommand\qedsymbol{$\blacksquare$}

\usepackage{amsthm}

\DeclareMathOperator*{\E}{\mathbb{E}}
\let\Pr\relax
\DeclareMathOperator*{\Pr}{\mathbb{P}}
\DeclareMathOperator*{\R}{\mathbb{E}}

\topmargin 0pt
\advance \topmargin by -\headheight
\advance \topmargin by -\headsep
\textheight 8.9in
\oddsidemargin 0pt
\evensidemargin \oddsidemargin
\marginparwidth 0.5in
\textwidth 6.5in

\parindent 0in
\parskip 1.5ex

\newcommand{\homework}[2]{
	\noindent
	\begin{center}
		\framebox{
			\vbox{
				\hbox to 6.50in { {\bf NYU Computer Science Bridge to Tandon Course} \hfill Winter 2021 }
				\vspace{4mm}
				\hbox to 6.50in { {\Large \hfill Homework 11  \hfill} }
				\vspace{2mm}
				\hbox to 6.50in { {Name: Danrong Li \hfill} NetID: dl4111}

			}
		}
	\end{center}
	\vspace*{4mm}
}

\begin{document}
	
	\homework{1}{Danrong Li}
	\section*{Question 5}
	
	\textbf{a. }Use mathematical induction to prove that for any positive integer n, 3 divide $n^3+2n$ leaving no remainder.
	\begin{proof}
	By induction on n.
	
	Base case: n = 1.
	
	$(n^3+2n)/3=3/3=1$, 3 divide 3 with no remainder, so the theorem holds for n = 1.
	
	Inductive step: Suppose that for positive integer k, 3 divide $k^2k$, then we will show that 3 divide $(k+1)^3+2(k+1)$. By induction hypothesis, 3 divide $k^3+2k$, which means $k^3+2k=3m$ for some integer m, which is equivalent to the induction hypothesis. We must show that $(k+1)^3+2(k+1)$ can be expressed as 3 time some integer.
	
	$(k+1)^3+2(k+1)$ = $k^2+3k^2+3k+1+2k+2$ Expand with the equation provided by the question
	
	\hspace{33.6mm}= $3m+3k^2+3k+3$ \hspace{10mm}Inductive Hypothesis $k^3+2k=3m$
	                 
	\hspace{33.6mm}= $3*(m+k^2+k+1)$ \hspace{7.5mm}By algebra
	                 
	k and m are both integers, so $(m+k^2+k+1)$ is also an integer. We can express $(k+1)^3+2(k+1)$ as 3 times an integer, which means that 3 divide $(k+1)^3+2(k+1)$. 
	
	Thus, for any positive integer n, 3 divide $n^3+2n$ leaving no remainder.
	\end{proof}
	
	\vspace{10mm}
	\textbf{b. }Use strong induction to prove that any positive integer n $(n\geq 2)$ can be written as a product of primes.
	
	\begin{proof}
	By strong induction on n.
	
	Base case: n = 2.
	
	n = 1 * 2, 1 and 2 are both prime numbers, so the theorem holds in the base case.
	
	Inductive step: Suppose that for any integer k, $k\geq 2$, for some integer j range from 2 to k, the integer j can be expressed as a product of primes. Then we will show that k+1 can be expressed as a product of primes. 
	
	If k+1 is prime number, then k+1 = 1 * (k+1), where 1 and k+1 are both prime numbers. Thus, the theorem hold true in this case.
	
	If k+1 is not prime number, then k+1 can be expressed as the product of 2 integers a, b, where a and b are both greater than or equal to 2. k+1 = a*b, a$\geq 2$, b$\geq 2$. k+1 = a*b, means a = (k+1)/b, since b $\geq 2$, we get (k+1)/b $<$ k+1, which means a $<$ k+1. With a being strictly less than k-1, we know that a$\leq$ k. In addition, we know that a$\leq$ k with the similar logic as shown above. Then, we have $2\leq a\leq k$, $2\leq b\leq k$. a and b both fall in the induction hypothesis criteria, so that they can be expresses as products of prime numbers. Since k+1 = a*b and we know that a and b can both be expressed as product of prime numbers, k+1 can also be expresses as product of prime numbers. 
	
	In the case where k+1 is prime number or the case where k+1 is not prime number, k+1 can be expressed as products of primes. 
	
	Thus, any positive integer n $(n\geq 2)$ can be written as a product of primes.
	
	\end{proof}
	
	\newpage
	\section*{Question 6}
	
	Solve the following questions from the Discrete Math Zybook:
	
	\textbf{1. }Exercise 7.4.1, sections a-g
	
	\textbf{a.}
	P(3) is true.
	
	$\sum_{j=1}^{3} j^2=1^2+2^2+3^2=1+4+9=14$
	
	$3*(3+1)*(2*3+1)/6=3*4*7/6=14$
	
	Thus, we verified P(3) is true.
	
	\textbf{b.}
	P(k)
	
	$\sum_{j=1}^{k} j^2=(k*(k+1)*(2k+1))/6$
	
	\textbf{c.}
	P(k+1)
	
	$\sum_{j=1}^{k+1} j^2=((k+1)*(k+1+1)*(2*(k+1)+1))/6$
	
	\textbf{d.}
	Base case, n = 1
	
	$\sum_{j=1}^{1} j^2=1$
	
	1*(1+1)*(2+1)/6 = 1
	
	so the theorem holds for the base case, n = 1
	
	\textbf{e.}
	inductive step to prove
	
	$\sum_{j=1}^{k+1} j^2=((k+1)*(k+1+1)*(2*(k+1)+1))/6$
	
	\textbf{f.}
	inductive hypothesis
	
	$\sum_{j=1}^{k} j^2=(k*(k+1)*(2k+1))/6$
	
	\textbf{g.}
	The entire proof is shown below:
	\begin{proof}
	By induction on n.
	
	Base case: n = 1.
	
	$\sum_{j=1}^{1} j^2=1$
	
	1*(1+1)*(2+1)/6 = 1
	
	so the theorem holds for the base case, n = 1
	
	Inductive step: Suppose that for any positive integer k, $k\geq 1$, $\sum_{j=1}^{k} j^2=(k*(k+1)*(2k+1))/6$, then we will show that $\sum_{j=1}^{k+1} j^2=((k+1)*(k+1+1)*(2*(k+1)+1))/6$. If we simplify this equation we get $\sum_{j=1}^{k+1} j^2=((k+1)(k+2)(2k+3))/6$, which is the induction hypothesis in the equivalent form.
	
	$\sum_{j=1}^{k+1} j^2=\sum_{j=1}^{k}+(k+1)^2$ Separate the last term

\hspace{13.5mm}	= $k(k+1)(2k+1)/6+(k+1)^2$ Induction Hypothesis

\hspace{13.5mm}	= $k(k+1)(2k+1)+6(k^2+1+2k)/6$ By algebra

\hspace{13.5mm}	= $(k+1)(k+2)(2k+3)/6$ By algebra

    We are able to show $\sum_{j=1}^{k+1} j^2=((k+1)(k+2)(2k+3))/6$. 
    
    Thus, for any positive integer n, $n\geq 1$, $\sum_{j=1}^{n} j^2=(n*(n+1)*(2n+1))/6$ 
	
	\end{proof}
	
	\vspace{20mm}
	\textbf{2. }Exercise 7.4.3, section c
	
	\textbf{c.}
	The entire proof is shown below:
	\begin{proof}
	By induction on n.
	
	Base case: n = 1.
	
	$\sum_{j=1}^1 1/j^2 = 1$
	
	2 - 1/1 = 1
	
	$1\leq 1$, so the theorem holds true in the base case.
	
	Inductive step: Suppose that for k$\geq 1$, $\sum_{j=1}^k 1/j^2 \leq 2-1/k$, then wen show that $\sum_{j=1}^{k+1} 1/j^2 \leq 2-1/(k+1)$
	
	$\sum_{j=1}^{k+1} 1/j^2$ = $\sum_{j=1}^k 1/j^2 + 1/(k+1)^2$ \hspace{2mm} Separate last term
	
\hspace{18mm}	$\leq 2-1/k+1/(k+1)^2$\hspace{8mm} Inductive Hypothesis
	
\hspace{18mm}	$\leq 2-1/k+1/(k(k+1))$ \hspace{3mm} By equations: $1/(k+1)^2\leq 1/(k(k+1))$, $k\geq 1$
	
\hspace{18mm}	= 2 + (1-(k+1))/(k(k+1))\hspace{2.5mm} By algebra
	
\hspace{18mm}	= 2 - 1/(k+1)\hspace{24mm} By algebra
	
	We have $\sum_{j=1}^{k+1} 1/j^2 \leq 2-1/(k+1)$, for k$\geq 1$.
	
	Thus, for n$\geq 1$, $\sum_{j=1}^n 1/j^2 \leq 2-1/n$
	
	\end{proof}
	
	
	
	\vspace{20mm}
	\textbf{3.} Exercise 7.5.1, section a
	
	\textbf{a.}
	Prove that for any positive integer n, 4 evenly divides $3^{2n}-1$
	
	\begin{proof}
	By induction on n.
	
	Base case: n = 1.
	
	$3^{2n}-1 = 3^2-1 = 8$
	
	8/4 = 2 with no remainder, so 4 evenly divides 8.
	
	The theorem holds true for the base case.
	
	Inductive step: Suppose that for positive integer k, 4 evenly divides $3^{2k}-1$, $3^{2k}-1=4m$, for some integer m. By adding 1 on both side, we get $3^{2k}=4m+1$, which is equivalent to inductive hypothesis. We will show that 4 evenly divides $3^{2(k+1)}-1$, which means we need to express $3^{2(k+1)}-1$ as 4 times some integer.
	
	$3^{2(k+1)}-1$ = $3^{2k+2}-1$\hspace{13.5mm} By algebra

\hspace{18.5mm} = $3^{2k}*3^2-1$\hspace{9.5mm} By algebra

\hspace{18.5mm} = $(4m+1)*3^2-1$ Inductive Hypothesis

\hspace{18.5mm} = 4 * m * 9 + 8\hspace{6mm} By algebra

\hspace{18.5mm} = 4 * (9 * m +2)\hspace{4.5mm} By algebra

Since m is an integer, 9m+2 is also an integer. We are able to express $3^{2(k+1)}-1$ as 4 times some integer, which means that 4 evenly divides $3^{2(k+1)}-1$

Thus, for any positive integer n, 4 evenly divides $3^{2n}-1$.


	\end{proof}
	

	
	
\end{document}