\documentclass[11pt]{article}
\usepackage{fullpage}
\usepackage{amsmath,amsfonts,amsthm,amssymb}
\usepackage{url}
\usepackage[demo]{graphicx}
\usepackage{caption} 
\usepackage{algpseudocode}
\usepackage{bbm}
\usepackage{float}
\usepackage{framed}
\usepackage{enumerate}
\usepackage{color}
\usepackage{mathtools}
\usepackage[colorlinks=true, linkcolor=red, urlcolor=blue, citecolor=blue]{hyperref}

\usepackage[utf8]{inputenc}
\usepackage[english]{babel}
\renewcommand\qedsymbol{$\blacksquare$}

\usepackage{amsthm}

\DeclareMathOperator*{\E}{\mathbb{E}}
\let\Pr\relax
\DeclareMathOperator*{\Pr}{\mathbb{P}}
\DeclareMathOperator*{\R}{\mathbb{E}}

\topmargin 0pt
\advance \topmargin by -\headheight
\advance \topmargin by -\headsep
\textheight 8.9in
\oddsidemargin 0pt
\evensidemargin \oddsidemargin
\marginparwidth 0.5in
\textwidth 6.5in

\parindent 0in
\parskip 1.5ex

\newcommand{\homework}[2]{
	\noindent
	\begin{center}
		\framebox{
			\vbox{
				\hbox to 6.50in { {\bf NYU Computer Science Bridge to Tandon Course} \hfill Winter 2021 }
				\vspace{4mm}
				\hbox to 6.50in { {\Large \hfill Homework 6  \hfill} }
				\vspace{2mm}
				\hbox to 6.50in { {Name: Danrong Li \hfill} NetID: dl4111}

			}
		}
	\end{center}
	\vspace*{4mm}
}

\begin{document}
	
	\homework{1}{Danrong Li}
	\section*{Question 5}
	
	Use the definition of $\Theta$ in order to show the following:
	
	\vspace{10mm}
	The definitions for $\Theta$, ${O}$ and $\Omega$ are as follows:
	
	1. T(N) = ${O}$(f(N)) 
	
	if there are positive constants $c_1$ and $n_0$ such that T(N) $\leq$ $c_1$*f(N) when N $\geq$ $n_0$.
	
	2. T(N) = $\Omega$(g(N))
	
	if there are positive constants $c_2$ and $n_0$ such that T(N) $\geq$ $c_2$*g(N) when N $\geq$ $n_0$.
	
	3. T(N) = $\Theta$(h(N))
	
	if and only if T(N) = ${O}$(h(N)) and T(N) = $\Omega$(h(N)).
	
	\vspace{10mm}
	\textbf{a.}
	$5n^3+2n^2+3n=\Theta (n^3)$
	
	If we take $c_1$ = 10, $c_2$ = 5, $n_0$ = 0,
	
	then for all $n\geq n_0$, we have:
	
	$5n^3+2n^2+3n\leq 5n^3+2n^3+3n^3=10n^3$
	
	$5n^3+2n^2+3n\geq 5n^3$
	
	Thus, we have:
	$5n^3\leq 5n^3+2n^2+3n\leq 10n^3$
	
	Therefore, according to the definitions stated above, we have
	
	$5n^3+2n^2+3n=\Theta (n^3)$
	
	\vspace{10mm}
	\textbf{b.}
	$\sqrt{7n^2+2n-8}=\Theta (n)$
	
	If we take $c_1$ = 3, $c_2$ = 1, $n_0$ = 1,
	
	then for all $n\geq n_0$, we have:
	
	$\sqrt{7n^2+2n-8}\leq \sqrt{7n^2+2n}\leq \sqrt{7n^2+2n^2}\leq \sqrt{9n^2}=3n$
	
	$\sqrt{7n^2+2n-8}\geq \sqrt{7n^2+2n-8n}\geq \sqrt{7n^2+(2-8)n^2}\geq \sqrt{n^2}=n$
	
	Thus, we have:
	$n\leq \sqrt{7n^2+2n-8}\leq 3n$
	
	Therefore, according to the definitions stated above, we have
	
	$\sqrt{7n^2+2n-8}=\Theta (n)$
	
	
\end{document}