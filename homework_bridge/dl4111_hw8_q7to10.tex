\documentclass[11pt]{article}
\usepackage{fullpage}
\usepackage{amsmath,amsfonts,amsthm,amssymb}
\usepackage{url}
\usepackage[demo]{graphicx}
\usepackage{caption} 
\usepackage{algpseudocode}
\usepackage{bbm}
\usepackage{float}
\usepackage{framed}
\usepackage{enumerate}
\usepackage{color}
\usepackage{mathtools}
\usepackage[colorlinks=true, linkcolor=red, urlcolor=blue, citecolor=blue]{hyperref}

\usepackage[utf8]{inputenc}
\usepackage[english]{babel}
\renewcommand\qedsymbol{$\blacksquare$}

\usepackage{amsthm}

\DeclareMathOperator*{\E}{\mathbb{E}}
\let\Pr\relax
\DeclareMathOperator*{\Pr}{\mathbb{P}}
\DeclareMathOperator*{\R}{\mathbb{E}}

\topmargin 0pt
\advance \topmargin by -\headheight
\advance \topmargin by -\headsep
\textheight 8.9in
\oddsidemargin 0pt
\evensidemargin \oddsidemargin
\marginparwidth 0.5in
\textwidth 6.5in

\parindent 0in
\parskip 1.5ex

\newcommand{\homework}[2]{
	\noindent
	\begin{center}
		\framebox{
			\vbox{
				\hbox to 6.50in { {\bf NYU Computer Science Bridge to Tandon Course} \hfill Winter 2021 }
				\vspace{4mm}
				\hbox to 6.50in { {\Large \hfill Homework 8  \hfill} }
				\vspace{2mm}
				\hbox to 6.50in { {Name: Danrong Li \hfill} NetID: dl4111}

			}
		}
	\end{center}
	\vspace*{4mm}
}

\begin{document}
	
	\homework{1}{Danrong Li}
	\section*{Question 7}
	
	Solve the following questions from the Discrete Math zyBook:
	
	\textbf{1. Exercise 6.1.5, section b-d}
	
	\textbf{b.}
	In three of a kind, 3 cards in the same rank and the other 2 cards has different ranks from the 3 cards and from each other. This means, the 3 cards can be chosen from 13 choices and the 2 cards would be to choose 2 from the remaining. In addition that there are 4 suits in a deck and the total number of options is to choose 5 from the 52 cards, we have the final answer:
	
    \textbf{13 * C(12,2) * C(4,3) * C(4,1) * C(4.1) / C(52,5)}
	
	\vspace{10mm}
	\textbf{c. }
	There are a total of 4 suits in a deck of card, and there are C(13,5) ways to choose 5 from 13 cards. In addition that the total number of options is to choose 5 from the 52 cards, we have the final answer:
	
	\textbf{C(4,1) * C(13,5) / C(52,5)}
	
	\vspace{10mm}
	\textbf{d.}
	In two of a kind, 2 cards have the same rank and the other 3 cards have different ranks from the 2 cards and from each other. In other words, the 2 cards can have 13 choices of ranks and the 3 cards can choose 3 from the 12 choices. n addition that there are 4 suits in a deck and the total number of options is to choose 5 from the 52 cards, we have the final answer:
	
	\textbf{13 * C(12,3) * C(4,2) * C(4,1) * C(4,1) * C(4,1) / C(52,5)}
	
	\vspace{10mm}
	\textbf{2. Exercise 6.2.4, sections a-d}
	
	\textbf{a.} 
	The probability of having at least one club is 1 - P(not have club). Excluding club, we have (52-13=) 39 cards. Thus we have the final answer:
	
	\textbf{1 - (C(39,5) / C(52,5))}
	
	\vspace{10mm}
	\textbf{b.}
	P(at least 2 cards with same rank) = 1 - P(no cards with same rank). We would first 5 ranks from 13 ranks and each rank has 4 choices in suit. 
	
	\textbf{Thus the final answer is:}
	$1-(((C(13,5))*((C(4,1))^5))/(C(52,5)))$
	
	\vspace{10mm}
	\textbf{c.}
	P(exactly 1 club or exactly 1 spade) = P(exactly 1 club) + P(exactly 1 spade) - P(exactly 1 club and exactly 1 spade). P(exactly 1 club) = P(exactly 1 spade) = choose 4 from (52-13=) 19 cards and the club could be from 13 choices. Similarly for both 1 club and 1 spade. Thus the final answer is:
	
	\textbf{(C(39,4) * 13 * 2 - C(26,3) * 13 * 13) / C(52,5)}

	
	\vspace{10mm}
	\textbf{d.}
	P(at least 2 clubs or at least 1 spade) = 1 - P(no club and no spade), thus the final answer is:
	
	\textbf{1 - C(26,5)/C(52,5)}
	
	\newpage
	\section*{Question 8}
	Solve the following questions from the Discrete Math zyBook:
	
	\textbf{1. Exercise 6.3.2, sections a-e}
	
	\textbf{a.}
	
	P(A) = P(b in the middle) = 6!/7! \textbf{= 1/7}
	
	P(B) = (6!+5*5!+5*4*4!+5*4*3*3!+5*4*3*2*2!+5*4*3*2*1)/(7!) \textbf{= 1/2}
	
	P(C) = P(def together) = 5!/7! \textbf{= 1/42}
	
	\vspace{10mm}
	\textbf{b.}
	$P(A|C)=P(A\cap C)/P(C)=|A\cap C|/|C|$ = 3!*2/(5!) \textbf{= 1/10}

	
	\vspace{10mm}
	\textbf{c.}
	$P(B|C)=P(B\cap C)/P(C)=|B\cap C|/|C|$ = 5!/2/(5!) \textbf{=1/2}
	
	This is because for every way to permute the letters with c being the right of b, there is exactly one way to permute with c being left of b. Then in exactly  half of the permutations, c being on the right of b.
	
	
	\vspace{10mm}
	\textbf{d.}
	$P(A|B)=P(A\cap B)/P(B)=|A\cap B|/|B$ = 6!/2/(7!/2) \textbf{= 1/7}

	The reasoning is similar to previous question.
	
	\vspace{10mm}
	\textbf{e.}
	
	P(A|B) = 1/7 = P(A)
	
	P(B|C) = 1/2 = P(B)
	
	P(A|C) = 1/10 != P(A)
	
	\textbf{Thus, A and B are independent; B and C are independent}
	

	\vspace{10mm}
	\textbf{2. Exercise 6.3.6, sections b,c}
	
	\textbf{b.}
	
	P(fist 5 H, last 5 T) = $(1/3)^5*(2/3)^5$
	
	\vspace{5mm}
	\textbf{c.}
	
	P(fist H, rest T) = $(1/3)*(2/3)^9$
	
	\vspace{10mm}
	\textbf{3. Exercise 6.4.2, section a}
	
	\textbf{a.}
	
	P(Fair) = P(Biased) = 1/2
	
	P(Observation | Fair) = $(1/6)^6$
	
	P(Observation | Biased) = $0.15^4*0.25^2$
	
	\textbf{Thus, with Bayes' theorem:}
	
	P(Fair | Observation) = $(1/6)^6/2/((1/6)^6/2+0.15^4*0.25^2/2)$ \textbf{= 0.4038}

	\newpage
	\section*{Question 9}
	Solve the following questions from the Discrete Math zyBook:
	
	\textbf{1. a.}
	\textbf{\{0,1,2,3,4\}}
	
	\vspace{10mm}
	\textbf{1. b.}
	
	(0, C(48,5)/C(52,5)), (1, 4*C(48,4)/C(52,5), (2, C(4,2)*C(48,3)/C(52,5)), (3, C(4,3)*C(48,2)/C(52,5)), (4, C(4,4)*C(48,1)/C(52,5)))
	
	\vspace{10mm}
	\textbf{2. a.}
	
	E[G] = 0 * C(3,2)/C(10,2) + 1 * C(7,1)*C(3,1)/C(10,2) + 2 * C(7,2)/C(10,2) \textbf{= 1.4}
	
	\vspace{10mm}
	\textbf{3. a.}
	
	E[X] = (0+1+4+9+16+25+36)/6 \textbf{= 91/6 = 15.17}
	
	\vspace{10mm}
	\textbf{3. b.}
	
	E[Y] = 0*1/8+1*3/8+4*3/8+9*1/8 \textbf{= 3}
	
	\vspace{10mm}
	\textbf{4. a.}
	
	E[S] = $E[S_1+S_2+...+S_{10}]=E[S_1]+E{S_2}+...+E[S_{10}]=10*E[S_1]$
	
	$E[S_1]=$1/10*1+9/10*0=1/10
	
	Thus, E[S] = 10*1/10 \textbf{= 1}

	
	\newpage
	\section*{Question 10}
	Solve the following questions from the Discrete Math zyBook:
	
	\textbf{1. a.}
	
	P(2 out of 100 have defects) = $C(100,2)*(0.01)^2*0.99^{98}$ \textbf{= 0.185}
	

	\vspace{10mm}
	\textbf{1. b.}
	
	P(at least 2 out 100 have defects) = $1-(C(100,0)*0.99^{100}+100/100*0.99^{99})$ \textbf{= 0.264}
	
	\vspace{10mm}
	\textbf{1. c.}
	
	E[D] = n*p = 100*0.01 \textbf{= 1}
	
	\vspace{10mm}
	\textbf{1. d.}
	
	P(at least 2 boards with Defect) = P(at least 1 batch with Defect) = 1 - P(no batch with Defect) = $1-C(50,0)*0.01^0*0.99^{50}$ \textbf{= 0.395} $>0.264$
	
	E[D] = n*p = 50 * 1/100 * 2 \textbf{= 1}
	
	\textbf{Thus, the P(at least 2 boards with Defect) in the batch situation is higher than the probability in the separate situation. And the expected number of circuit boards with Defect is the same in both situations.}
	
	\vspace{10mm}
	\textbf{2. b.}
	The conclusion of the coin is fair is made when we have at least 4 Heads. P(H $\geq$ 4) = 1 - P(H $<$ 4), thus we have the following equation: 
	
	$1-(C(10,0)*0.3^0*0.7^{10}+C(10,1)*0.3*0.7^9+C(10,2)*0.3^2*0.7^8+C(10,3)*0.3^3*0.7^7)$ \textbf{= 0.3504}
	
	
\end{document}