\documentclass[11pt]{article}
\usepackage{fullpage}
\usepackage{amsmath,amsfonts,amsthm,amssymb}
\usepackage{url}
\usepackage[demo]{graphicx}
\usepackage{caption} 
\usepackage{algpseudocode}
\usepackage{bbm}
\usepackage{float}
\usepackage{framed}
\usepackage{enumerate}
\usepackage{color}
\usepackage{mathtools}
\usepackage[colorlinks=true, linkcolor=red, urlcolor=blue, citecolor=blue]{hyperref}

\usepackage[utf8]{inputenc}
\usepackage[english]{babel}
\renewcommand\qedsymbol{$\blacksquare$}

\usepackage{amsthm}

\DeclareMathOperator*{\E}{\mathbb{E}}
\let\Pr\relax
\DeclareMathOperator*{\Pr}{\mathbb{P}}
\DeclareMathOperator*{\R}{\mathbb{E}}

\topmargin 0pt
\advance \topmargin by -\headheight
\advance \topmargin by -\headsep
\textheight 8.9in
\oddsidemargin 0pt
\evensidemargin \oddsidemargin
\marginparwidth 0.5in
\textwidth 6.5in

\parindent 0in
\parskip 1.5ex

\newcommand{\homework}[2]{
	\noindent
	\begin{center}
		\framebox{
			\vbox{
				\hbox to 6.50in { {\bf NYU Computer Science Bridge to Tandon Course} \hfill Winter 2021 }
				\vspace{4mm}
				\hbox to 6.50in { {\Large \hfill Homework 7  \hfill} }
				\vspace{2mm}
				\hbox to 6.50in { {Name: Danrong Li \hfill} NetID: dl4111}

			}
		}
	\end{center}
	\vspace*{4mm}
}

\begin{document}
	
	\homework{1}{Danrong Li}
	\section*{Question 3}
	
	\textbf{1. Solve Exercise 8.2.2, section b from the Discrete Book.}
	
	\vspace{10mm}
	The definitions for $\Theta$, ${O}$ and $\Omega$ are as follows:
	
	1. T(N) = ${O}$(f(N)) 
	
	if there are positive constants $c_1$ and $n_0$ such that T(N) $\leq$ $c_1$*f(N) when N $\geq$ $n_0$.
	
	2. T(N) = $\Omega$(g(N))
	
	if there are positive constants $c_2$ and $n_0$ such that T(N) $\geq$ $c_2$*g(N) when N $\geq$ $n_0$.
	
	3. T(N) = $\Theta$(h(N))
	
	if and only if T(N) = ${O}$(h(N)) and T(N) = $\Omega$(h(N)).
	
	\vspace{10mm}
	\textbf{b. }
	$n^3+3n^2+4=\Theta (n^3)$
	
	We first prove $n^3+3n^2+4={O}(n^3)$
	\begin{proof}
    If we take $c_1$ = 8, $n_0$ = 1,
	
	then for all $n\geq n_0$, we have:
	
	$n^3+3n^2+4\leq n^3+3n^3+4n^3\leq 8n^3$
	
	The above equation makes sense since we have $n\geq n_0$.
	
	And it proves that for $n\geq n_0$, $n^3+3n^2+4\leq 8n^3$, which means $n^3+3n^2+4={O}(n^3)$
    \end{proof}

	We then prove that $n^3+3n^2+4=\Omega(n^3)$
	\begin{proof}
	If we take $c_2$ = 1, $n_0$ = 0,
	
	then for all $n\geq n_0$, we have:
	
	$n^3+3n^2+4\geq n^3+3n^2\geq n^3$
	
	The above equation makes sense since we have $n\geq n_0$.
	
	And it proves that $n^3+3n^2+4\geq n^3$, which means $n^3+3n^2+4=\Omega(n^3)$
	\end{proof}
	
	\textbf{Therefore, according to the definitions stated above, we can prove:}
	
	$n^3+3n^2+4=\Theta (n^3)$
	
	\vspace{10mm}
	\textbf{2. Solve Exercise 8.3.5, sections a-e from the Discrete Math zyBook}
	
	\newpage
	\section*{Question 4}
	Solve the following questions from the Discrete Math zyBook:
	
	\textbf{1. b}
	
	\vspace{10mm}
	\textbf{1. c}
	
	\vspace{10mm}
	\textbf{2. a}
	
	\vspace{10mm}
	\textbf{3. b}
	
	\vspace{10mm}
	\textbf{3. c}
	
	\vspace{10mm}
	\textbf{4. a}
	
	\vspace{10mm}
	\textbf{4. b}
	
	\newpage
	\section*{Question 5}
	Solve the following questions from the Discrete Math zyBook:
	
	\textbf{1. a}
	
	
	\vspace{10mm}
	\textbf{1. b}
	
	\vspace{10mm}
	\textbf{2. a}
	
	\vspace{10mm}
	\textbf{2. b}
	
	\vspace{10mm}
	\textbf{2. c}
	
	\vspace{10mm}
	\textbf{2. d}
	
	\vspace{10mm}
	\textbf{2. e}
	
	\vspace{10mm}
	\textbf{2. f}
	
	\vspace{10mm}
	\textbf{2. g}
	
	
	\vspace{10mm}
	\textbf{3. a}
	
	\vspace{10mm}
	\textbf{4. c}
	
	\vspace{10mm}
	\textbf{4. d}
	
	\vspace{10mm}
	\textbf{4. e}
	
	\vspace{10mm}
	\textbf{4. f}
	
	\vspace{10mm}
	\textbf{5. a}
	
	\vspace{10mm}
	\textbf{5. b}

	
	\newpage
	\section*{Question 6}
	Solve the following questions from the Discrete Math zyBook:
	
	\textbf{1. a}
	
	\vspace{10mm}
	\textbf{1. b}
	
	\vspace{10mm}
	\textbf{2. a}
	
	\vspace{10mm}
	\textbf{2. b}
	
	\newpage
	\section*{Question 7}
	How many one-to-one functions are there from a set with five elements to sets with the following number of elements?
	
	\textbf{a. 4}
	
	\textbf{b. 5}
	
	\textbf{c. 6}
	
	\textbf{d. 7}
	
	
\end{document}